% Options for packages loaded elsewhere
\PassOptionsToPackage{unicode}{hyperref}
\PassOptionsToPackage{hyphens}{url}
%
\documentclass[
]{article}
\usepackage{amsmath,amssymb}
\usepackage{iftex}
\ifPDFTeX
  \usepackage[T1]{fontenc}
  \usepackage[utf8]{inputenc}
  \usepackage{textcomp} % provide euro and other symbols
\else % if luatex or xetex
  \usepackage{unicode-math} % this also loads fontspec
  \defaultfontfeatures{Scale=MatchLowercase}
  \defaultfontfeatures[\rmfamily]{Ligatures=TeX,Scale=1}
\fi
\usepackage{lmodern}
\ifPDFTeX\else
  % xetex/luatex font selection
\fi
% Use upquote if available, for straight quotes in verbatim environments
\IfFileExists{upquote.sty}{\usepackage{upquote}}{}
\IfFileExists{microtype.sty}{% use microtype if available
  \usepackage[]{microtype}
  \UseMicrotypeSet[protrusion]{basicmath} % disable protrusion for tt fonts
}{}
\makeatletter
\@ifundefined{KOMAClassName}{% if non-KOMA class
  \IfFileExists{parskip.sty}{%
    \usepackage{parskip}
  }{% else
    \setlength{\parindent}{0pt}
    \setlength{\parskip}{6pt plus 2pt minus 1pt}}
}{% if KOMA class
  \KOMAoptions{parskip=half}}
\makeatother
\usepackage{xcolor}
\usepackage[margin=1in]{geometry}
\usepackage{color}
\usepackage{fancyvrb}
\newcommand{\VerbBar}{|}
\newcommand{\VERB}{\Verb[commandchars=\\\{\}]}
\DefineVerbatimEnvironment{Highlighting}{Verbatim}{commandchars=\\\{\}}
% Add ',fontsize=\small' for more characters per line
\usepackage{framed}
\definecolor{shadecolor}{RGB}{248,248,248}
\newenvironment{Shaded}{\begin{snugshade}}{\end{snugshade}}
\newcommand{\AlertTok}[1]{\textcolor[rgb]{0.94,0.16,0.16}{#1}}
\newcommand{\AnnotationTok}[1]{\textcolor[rgb]{0.56,0.35,0.01}{\textbf{\textit{#1}}}}
\newcommand{\AttributeTok}[1]{\textcolor[rgb]{0.13,0.29,0.53}{#1}}
\newcommand{\BaseNTok}[1]{\textcolor[rgb]{0.00,0.00,0.81}{#1}}
\newcommand{\BuiltInTok}[1]{#1}
\newcommand{\CharTok}[1]{\textcolor[rgb]{0.31,0.60,0.02}{#1}}
\newcommand{\CommentTok}[1]{\textcolor[rgb]{0.56,0.35,0.01}{\textit{#1}}}
\newcommand{\CommentVarTok}[1]{\textcolor[rgb]{0.56,0.35,0.01}{\textbf{\textit{#1}}}}
\newcommand{\ConstantTok}[1]{\textcolor[rgb]{0.56,0.35,0.01}{#1}}
\newcommand{\ControlFlowTok}[1]{\textcolor[rgb]{0.13,0.29,0.53}{\textbf{#1}}}
\newcommand{\DataTypeTok}[1]{\textcolor[rgb]{0.13,0.29,0.53}{#1}}
\newcommand{\DecValTok}[1]{\textcolor[rgb]{0.00,0.00,0.81}{#1}}
\newcommand{\DocumentationTok}[1]{\textcolor[rgb]{0.56,0.35,0.01}{\textbf{\textit{#1}}}}
\newcommand{\ErrorTok}[1]{\textcolor[rgb]{0.64,0.00,0.00}{\textbf{#1}}}
\newcommand{\ExtensionTok}[1]{#1}
\newcommand{\FloatTok}[1]{\textcolor[rgb]{0.00,0.00,0.81}{#1}}
\newcommand{\FunctionTok}[1]{\textcolor[rgb]{0.13,0.29,0.53}{\textbf{#1}}}
\newcommand{\ImportTok}[1]{#1}
\newcommand{\InformationTok}[1]{\textcolor[rgb]{0.56,0.35,0.01}{\textbf{\textit{#1}}}}
\newcommand{\KeywordTok}[1]{\textcolor[rgb]{0.13,0.29,0.53}{\textbf{#1}}}
\newcommand{\NormalTok}[1]{#1}
\newcommand{\OperatorTok}[1]{\textcolor[rgb]{0.81,0.36,0.00}{\textbf{#1}}}
\newcommand{\OtherTok}[1]{\textcolor[rgb]{0.56,0.35,0.01}{#1}}
\newcommand{\PreprocessorTok}[1]{\textcolor[rgb]{0.56,0.35,0.01}{\textit{#1}}}
\newcommand{\RegionMarkerTok}[1]{#1}
\newcommand{\SpecialCharTok}[1]{\textcolor[rgb]{0.81,0.36,0.00}{\textbf{#1}}}
\newcommand{\SpecialStringTok}[1]{\textcolor[rgb]{0.31,0.60,0.02}{#1}}
\newcommand{\StringTok}[1]{\textcolor[rgb]{0.31,0.60,0.02}{#1}}
\newcommand{\VariableTok}[1]{\textcolor[rgb]{0.00,0.00,0.00}{#1}}
\newcommand{\VerbatimStringTok}[1]{\textcolor[rgb]{0.31,0.60,0.02}{#1}}
\newcommand{\WarningTok}[1]{\textcolor[rgb]{0.56,0.35,0.01}{\textbf{\textit{#1}}}}
\usepackage{graphicx}
\makeatletter
\def\maxwidth{\ifdim\Gin@nat@width>\linewidth\linewidth\else\Gin@nat@width\fi}
\def\maxheight{\ifdim\Gin@nat@height>\textheight\textheight\else\Gin@nat@height\fi}
\makeatother
% Scale images if necessary, so that they will not overflow the page
% margins by default, and it is still possible to overwrite the defaults
% using explicit options in \includegraphics[width, height, ...]{}
\setkeys{Gin}{width=\maxwidth,height=\maxheight,keepaspectratio}
% Set default figure placement to htbp
\makeatletter
\def\fps@figure{htbp}
\makeatother
\setlength{\emergencystretch}{3em} % prevent overfull lines
\providecommand{\tightlist}{%
  \setlength{\itemsep}{0pt}\setlength{\parskip}{0pt}}
\setcounter{secnumdepth}{-\maxdimen} % remove section numbering
\ifLuaTeX
  \usepackage{selnolig}  % disable illegal ligatures
\fi
\IfFileExists{bookmark.sty}{\usepackage{bookmark}}{\usepackage{hyperref}}
\IfFileExists{xurl.sty}{\usepackage{xurl}}{} % add URL line breaks if available
\urlstyle{same}
\hypersetup{
  pdftitle={WineQualityChallenge},
  pdfauthor={Luca Marchiori},
  hidelinks,
  pdfcreator={LaTeX via pandoc}}

\title{WineQualityChallenge}
\author{Luca Marchiori}
\date{2024-03-27}

\begin{document}
\maketitle

\hypertarget{wine-quality-challenge}{%
\section{Wine Quality Challenge}\label{wine-quality-challenge}}

\hypertarget{instructions}{%
\subsection{Instructions}\label{instructions}}

The inputs include objective tests (e.g.~PH values) and the output is
based on sensory data (median of at least 3 evaluations made by wine
experts). Each expert graded the wine quality between 0 (very bad) and
10 (very excellent).

Submissions are evaluated on Root-Mean-Squared-Error (RMSE) between the
predicted value and the observed quality.

RMSE = sqrt(mean((y -- haty)\^{}2))

\hypertarget{dataset-colums}{%
\subsection{Dataset colums}\label{dataset-colums}}

\begin{enumerate}
\def\labelenumi{\arabic{enumi}.}
\tightlist
\item
  fixed acidity
\item
  volatile acidity
\item
  citric acid
\item
  residual sugar
\item
  chlorides
\item
  free sulfur dioxide
\item
  total sulfur dioxide
\item
  density
\item
  pH
\item
  sulphates
\item
  alcohol
\item
  quality (score between 0 and 10) \#\# Data import
\end{enumerate}

\begin{Shaded}
\begin{Highlighting}[]
\CommentTok{\# Load the training data}
\NormalTok{train }\OtherTok{\textless{}{-}} \FunctionTok{read.csv}\NormalTok{(}\StringTok{"wineq\_train.csv"}\NormalTok{, }\AttributeTok{stringsAsFactors=}\NormalTok{F)}
\CommentTok{\# Load the test data}
\NormalTok{test }\OtherTok{\textless{}{-}} \FunctionTok{read.csv}\NormalTok{(}\StringTok{"wineq\_validation.csv"}\NormalTok{, , }\AttributeTok{stringsAsFactors=}\NormalTok{F)}
\end{Highlighting}
\end{Shaded}

\hypertarget{exploratory-data-analysis}{%
\subsection{Exploratory Data Analysis}\label{exploratory-data-analysis}}

\begin{Shaded}
\begin{Highlighting}[]
\CommentTok{\# Compactly Display the Structure of an Arbitrary R Object}
\FunctionTok{str}\NormalTok{(train)}
\end{Highlighting}
\end{Shaded}

\begin{verbatim}
## 'data.frame':    3698 obs. of  12 variables:
##  $ fixed.acidity       : num  7 6.3 8.1 7.2 7.2 8.1 7 8.1 8.1 8.6 ...
##  $ volatile.acidity    : num  0.27 0.3 0.28 0.23 0.23 0.28 0.27 0.22 0.27 0.23 ...
##  $ citric.acid         : num  0.36 0.34 0.4 0.32 0.32 0.4 0.36 0.43 0.41 0.4 ...
##  $ residual.sugar      : num  20.7 1.6 6.9 8.5 8.5 6.9 20.7 1.5 1.45 4.2 ...
##  $ chlorides           : num  0.045 0.049 0.05 0.058 0.058 0.05 0.045 0.044 0.033 0.035 ...
##  $ free.sulfur.dioxide : num  45 14 30 47 47 30 45 28 11 17 ...
##  $ total.sulfur.dioxide: num  170 132 97 186 186 97 170 129 63 109 ...
##  $ density             : num  1.001 0.994 0.995 0.996 0.996 ...
##  $ pH                  : num  3 3.3 3.26 3.19 3.19 3.26 3 3.22 2.99 3.14 ...
##  $ sulphates           : num  0.45 0.49 0.44 0.4 0.4 0.44 0.45 0.45 0.56 0.53 ...
##  $ alcohol             : num  8.8 9.5 10.1 9.9 9.9 10.1 8.8 11 12 9.7 ...
##  $ quality             : int  6 6 6 6 6 6 6 6 5 5 ...
\end{verbatim}

\begin{Shaded}
\begin{Highlighting}[]
\CommentTok{\# Take an initial view of the dataset}
\FunctionTok{summary}\NormalTok{(train)}
\end{Highlighting}
\end{Shaded}

\begin{verbatim}
##  fixed.acidity   volatile.acidity  citric.acid     residual.sugar  
##  Min.   : 4.20   Min.   :0.0800   Min.   :0.0000   Min.   : 0.600  
##  1st Qu.: 6.30   1st Qu.:0.2100   1st Qu.:0.2700   1st Qu.: 1.800  
##  Median : 6.80   Median :0.2600   Median :0.3200   Median : 5.200  
##  Mean   : 6.86   Mean   :0.2791   Mean   :0.3348   Mean   : 6.441  
##  3rd Qu.: 7.30   3rd Qu.:0.3200   3rd Qu.:0.3900   3rd Qu.:10.000  
##  Max.   :14.20   Max.   :1.1000   Max.   :1.0000   Max.   :65.800  
##    chlorides       free.sulfur.dioxide total.sulfur.dioxide    density      
##  Min.   :0.01200   Min.   :  2.00      Min.   :  9.0        Min.   :0.9871  
##  1st Qu.:0.03600   1st Qu.: 23.00      1st Qu.:108.0        1st Qu.:0.9917  
##  Median :0.04300   Median : 34.00      Median :134.0        Median :0.9938  
##  Mean   :0.04562   Mean   : 35.51      Mean   :138.6        Mean   :0.9940  
##  3rd Qu.:0.05000   3rd Qu.: 46.00      3rd Qu.:168.0        3rd Qu.:0.9962  
##  Max.   :0.29000   Max.   :289.00      Max.   :440.0        Max.   :1.0390  
##        pH          sulphates         alcohol         quality     
##  Min.   :2.720   Min.   :0.2200   Min.   : 8.40   Min.   :3.000  
##  1st Qu.:3.090   1st Qu.:0.4100   1st Qu.: 9.40   1st Qu.:5.000  
##  Median :3.180   Median :0.4700   Median :10.40   Median :6.000  
##  Mean   :3.186   Mean   :0.4899   Mean   :10.52   Mean   :5.879  
##  3rd Qu.:3.280   3rd Qu.:0.5500   3rd Qu.:11.40   3rd Qu.:6.000  
##  Max.   :3.820   Max.   :1.0800   Max.   :14.20   Max.   :9.000
\end{verbatim}

\hypertarget{fit-a-basic-linear-model}{%
\subsubsection{Fit a basic linear
model}\label{fit-a-basic-linear-model}}

The lm function is used to create linear models.
\texttt{quality\ \textasciitilde{}\ .} is the formula provided to the
lm() function. In R modeling formulas, the tilde \textasciitilde{}
separates the outcome variable from the predictor variables. In this
case, quality is the outcome variable, and . means ``all other variables
in the data frame''. So essentially, it's saying ``predict quality using
all other variables in the train data frame''.

\texttt{data=train} specifies the data frame to be used for fitting the
model. In this case, it's using the train data frame.

\texttt{predict()} is a function in R used to generate predictions from
various types of models, including linear regression models. The
argument \texttt{newdata=test} specifies the data frame (test) for which
we want to predict the outcome quality variable.

In summary, use the linear regression model (\texttt{fit}) to predict
the outcome variable for the observations in the test data frame, and
store these predictions in the variable \texttt{yhat}.

\begin{Shaded}
\begin{Highlighting}[]
\NormalTok{fit }\OtherTok{=} \FunctionTok{lm}\NormalTok{(quality }\SpecialCharTok{\textasciitilde{}}\NormalTok{ ., }\AttributeTok{data=}\NormalTok{train)}
\FunctionTok{summary}\NormalTok{(fit)}
\end{Highlighting}
\end{Shaded}

\begin{verbatim}
## 
## Call:
## lm(formula = quality ~ ., data = train)
## 
## Residuals:
##     Min      1Q  Median      3Q     Max 
## -3.8520 -0.4980 -0.0442  0.4684  3.0817 
## 
## Coefficients:
##                        Estimate Std. Error t value Pr(>|t|)    
## (Intercept)           1.337e+02  2.075e+01   6.446 1.30e-10 ***
## fixed.acidity         6.402e-02  2.370e-02   2.701  0.00694 ** 
## volatile.acidity     -1.853e+00  1.300e-01 -14.259  < 2e-16 ***
## citric.acid           7.465e-02  1.126e-01   0.663  0.50724    
## residual.sugar        7.516e-02  8.423e-03   8.923  < 2e-16 ***
## chlorides            -9.156e-01  6.665e-01  -1.374  0.16958    
## free.sulfur.dioxide   3.971e-03  9.751e-04   4.072 4.75e-05 ***
## total.sulfur.dioxide -4.050e-04  4.421e-04  -0.916  0.35966    
## density              -1.338e+02  2.106e+01  -6.353 2.36e-10 ***
## pH                    6.782e-01  1.210e-01   5.606 2.23e-08 ***
## sulphates             5.740e-01  1.144e-01   5.019 5.44e-07 ***
## alcohol               2.086e-01  2.685e-02   7.768 1.03e-14 ***
## ---
## Signif. codes:  0 '***' 0.001 '**' 0.01 '*' 0.05 '.' 0.1 ' ' 1
## 
## Residual standard error: 0.7544 on 3686 degrees of freedom
## Multiple R-squared:  0.2814, Adjusted R-squared:  0.2792 
## F-statistic: 131.2 on 11 and 3686 DF,  p-value: < 2.2e-16
\end{verbatim}

\begin{Shaded}
\begin{Highlighting}[]
\NormalTok{yhat }\OtherTok{=} \FunctionTok{predict}\NormalTok{(fit, }\AttributeTok{newdata=}\NormalTok{test)}

\CommentTok{\# Put the prediction in a file for the submission on the platform}
\FunctionTok{write.table}\NormalTok{(}\AttributeTok{file=}\StringTok{"mySubmission.txt"}\NormalTok{, yhat, }\AttributeTok{row.names =} \ConstantTok{FALSE}\NormalTok{, }\AttributeTok{col.names =} \ConstantTok{FALSE}\NormalTok{)}
\end{Highlighting}
\end{Shaded}

\hypertarget{computing-minimal-squared-error}{%
\subsection{Computing Minimal Squared
Error}\label{computing-minimal-squared-error}}

MSE (Mean Squared Error) is a common metric for evaluating the
performance of regression models, by comparing the observed values
\texttt{y} with the predicted values \texttt{yhat} generated by the
provided \texttt{model}.

\begin{Shaded}
\begin{Highlighting}[]
\NormalTok{MSE }\OtherTok{\textless{}{-}} \ControlFlowTok{function}\NormalTok{(y, model)\{}
\NormalTok{  yhat }\OtherTok{=} \FunctionTok{predict}\NormalTok{(model) }\CommentTok{\# Estimated y computed on the basis of the features}
  \FunctionTok{mean}\NormalTok{((y}\SpecialCharTok{{-}}\NormalTok{yhat)}\SpecialCharTok{\^{}}\DecValTok{2}\NormalTok{)}
\NormalTok{\}}

\NormalTok{SQRTMSE }\OtherTok{\textless{}{-}} \ControlFlowTok{function}\NormalTok{(y, model)\{}
  \FunctionTok{sqrt}\NormalTok{(}\FunctionTok{MSE}\NormalTok{(y,model))}
\NormalTok{\}}
\end{Highlighting}
\end{Shaded}

\hypertarget{get-mse-for-basic-linear-model}{%
\subsection{Get MSE for basic linear
model}\label{get-mse-for-basic-linear-model}}

\begin{Shaded}
\begin{Highlighting}[]
\NormalTok{mse }\OtherTok{=} \FunctionTok{MSE}\NormalTok{(}\AttributeTok{y=}\NormalTok{train}\SpecialCharTok{$}\NormalTok{quality, }\AttributeTok{model=}\NormalTok{fit)}
\CommentTok{\# Square root of MSE}
\FunctionTok{sqrt}\NormalTok{(mse)}
\end{Highlighting}
\end{Shaded}

\begin{verbatim}
## [1] 0.7531519
\end{verbatim}

\begin{Shaded}
\begin{Highlighting}[]
\CommentTok{\#library(ggplot2)}
\CommentTok{\#require(graphics)}
\CommentTok{\#pairs(train)}
\CommentTok{\#ggplot(cor(train))}

\CommentTok{\#ggcorrplot(cor(train))}
\CommentTok{\#heatmap(cor(train))}

\CommentTok{\#plot(train$alcohol, train$quality)}
\CommentTok{\#plot(train$total.sulfur.dioxide, train$free.sulfur.dioxide)}
\CommentTok{\#abline(fit)}
\CommentTok{\#cor(train)}
\end{Highlighting}
\end{Shaded}

\hypertarget{optimizing-linear-model}{%
\subsection{Optimizing linear model}\label{optimizing-linear-model}}

Create a new linear model without considering citric.acid, chlorides and
total.sulfur.dioxide We get a SQMSE of 0.7534613 which is better

\begin{Shaded}
\begin{Highlighting}[]
\NormalTok{fit\_m3 }\OtherTok{=} \FunctionTok{lm}\NormalTok{(quality }\SpecialCharTok{\textasciitilde{}}\NormalTok{ . }\SpecialCharTok{{-}}\NormalTok{citric.acid }\SpecialCharTok{{-}}\NormalTok{chlorides }\SpecialCharTok{{-}}\NormalTok{total.sulfur.dioxide, }\AttributeTok{data=}\NormalTok{train)}
\NormalTok{yhat }\OtherTok{=} \FunctionTok{predict}\NormalTok{(fit\_m3, }\AttributeTok{newdata=}\NormalTok{test)}
\FunctionTok{SQRTMSE}\NormalTok{(}\AttributeTok{y=}\NormalTok{train}\SpecialCharTok{$}\NormalTok{quality, }\AttributeTok{model=}\NormalTok{fit\_m3)}
\end{Highlighting}
\end{Shaded}

\begin{verbatim}
## [1] 0.7534613
\end{verbatim}

\begin{Shaded}
\begin{Highlighting}[]
\FunctionTok{write.table}\NormalTok{(}\AttributeTok{file=}\StringTok{"mySubmission.txt"}\NormalTok{, yhat, }\AttributeTok{row.names =} \ConstantTok{FALSE}\NormalTok{, }\AttributeTok{col.names =} \ConstantTok{FALSE}\NormalTok{)}
\end{Highlighting}
\end{Shaded}

Taking out also density, leads to a SQRTMSE of 0.7583856 which is worse

\begin{Shaded}
\begin{Highlighting}[]
\NormalTok{fit\_m3n2 }\OtherTok{=} \FunctionTok{lm}\NormalTok{(quality }\SpecialCharTok{\textasciitilde{}}\NormalTok{ . }\SpecialCharTok{{-}}\NormalTok{citric.acid }\SpecialCharTok{{-}}\NormalTok{ chlorides }\SpecialCharTok{{-}}\NormalTok{ total.sulfur.dioxide }\SpecialCharTok{{-}}\NormalTok{ density, }\AttributeTok{data=}\NormalTok{train) }
\NormalTok{yhat }\OtherTok{=} \FunctionTok{predict}\NormalTok{(fit\_m3n2, }\AttributeTok{newdata=}\NormalTok{test)}
\FunctionTok{SQRTMSE}\NormalTok{(}\AttributeTok{y=}\NormalTok{train}\SpecialCharTok{$}\NormalTok{quality, }\AttributeTok{model=}\NormalTok{fit\_m3n2)}
\end{Highlighting}
\end{Shaded}

\begin{verbatim}
## [1] 0.7583856
\end{verbatim}

Taking out residual sugar but not density, leads to a SQRTMSE of
0.7627304 which is worse

\begin{Shaded}
\begin{Highlighting}[]
\NormalTok{fit\_m3n3 }\OtherTok{=} \FunctionTok{lm}\NormalTok{(quality }\SpecialCharTok{\textasciitilde{}}\NormalTok{ . }\SpecialCharTok{{-}}\NormalTok{citric.acid }\SpecialCharTok{{-}}\NormalTok{ chlorides }\SpecialCharTok{{-}}\NormalTok{ total.sulfur.dioxide }\SpecialCharTok{{-}}\NormalTok{ residual.sugar, }\AttributeTok{data=}\NormalTok{train) }
\NormalTok{yhat }\OtherTok{=} \FunctionTok{predict}\NormalTok{(fit\_m3n3, }\AttributeTok{newdata=}\NormalTok{test)}
\FunctionTok{SQRTMSE}\NormalTok{(}\AttributeTok{y=}\NormalTok{train}\SpecialCharTok{$}\NormalTok{quality, }\AttributeTok{model=}\NormalTok{fit\_m3n3)}
\end{Highlighting}
\end{Shaded}

\begin{verbatim}
## [1] 0.7627304
\end{verbatim}

\begin{Shaded}
\begin{Highlighting}[]
\CommentTok{\#fit\_m3n4 = lm(quality \textasciitilde{} . {-}train[,3] {-}train[,5] {-}train[,6] {-}train[,4], data=train) }
\CommentTok{\#yhat = predict(fit\_m3n4, newdata=test)}
\CommentTok{\#MSE(y=train$quality, model=fit\_m3n4)}
\end{Highlighting}
\end{Shaded}

Use only some other variables leads to a better SQRTMSE of 0.7612425

\begin{Shaded}
\begin{Highlighting}[]
\NormalTok{fit\_m3n4 }\OtherTok{=} \FunctionTok{lm}\NormalTok{(quality }\SpecialCharTok{\textasciitilde{}} \SpecialCharTok{+}\NormalTok{fixed.acidity }\SpecialCharTok{+}\NormalTok{ volatile.acidity }\SpecialCharTok{+}\NormalTok{ citric.acid }\SpecialCharTok{+}\NormalTok{ chlorides }\SpecialCharTok{+}\NormalTok{ free.sulfur.dioxide }\SpecialCharTok{+}\NormalTok{ total.sulfur.dioxide }\SpecialCharTok{+}\NormalTok{ density }\SpecialCharTok{+}\NormalTok{ pH }\SpecialCharTok{+}\NormalTok{ sulphates }\SpecialCharTok{+}\NormalTok{ alcohol, }\AttributeTok{data=}\NormalTok{train) }
\NormalTok{yhat }\OtherTok{=} \FunctionTok{predict}\NormalTok{(fit\_m3n4, }\AttributeTok{newdata=}\NormalTok{test)}
\FunctionTok{SQRTMSE}\NormalTok{(}\AttributeTok{y=}\NormalTok{train}\SpecialCharTok{$}\NormalTok{quality, }\AttributeTok{model=}\NormalTok{fit\_m3n4)}
\end{Highlighting}
\end{Shaded}

\begin{verbatim}
## [1] 0.7612425
\end{verbatim}

\begin{Shaded}
\begin{Highlighting}[]
\FunctionTok{write.table}\NormalTok{(}\AttributeTok{file=}\StringTok{"mySubmission.txt"}\NormalTok{, yhat, }\AttributeTok{row.names =} \ConstantTok{FALSE}\NormalTok{, }\AttributeTok{col.names =} \ConstantTok{FALSE}\NormalTok{)}
\end{Highlighting}
\end{Shaded}


\end{document}
