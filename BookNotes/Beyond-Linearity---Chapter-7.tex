% Options for packages loaded elsewhere
\PassOptionsToPackage{unicode}{hyperref}
\PassOptionsToPackage{hyphens}{url}
%
\documentclass[
]{article}
\usepackage{amsmath,amssymb}
\usepackage{iftex}
\ifPDFTeX
  \usepackage[T1]{fontenc}
  \usepackage[utf8]{inputenc}
  \usepackage{textcomp} % provide euro and other symbols
\else % if luatex or xetex
  \usepackage{unicode-math} % this also loads fontspec
  \defaultfontfeatures{Scale=MatchLowercase}
  \defaultfontfeatures[\rmfamily]{Ligatures=TeX,Scale=1}
\fi
\usepackage{lmodern}
\ifPDFTeX\else
  % xetex/luatex font selection
\fi
% Use upquote if available, for straight quotes in verbatim environments
\IfFileExists{upquote.sty}{\usepackage{upquote}}{}
\IfFileExists{microtype.sty}{% use microtype if available
  \usepackage[]{microtype}
  \UseMicrotypeSet[protrusion]{basicmath} % disable protrusion for tt fonts
}{}
\makeatletter
\@ifundefined{KOMAClassName}{% if non-KOMA class
  \IfFileExists{parskip.sty}{%
    \usepackage{parskip}
  }{% else
    \setlength{\parindent}{0pt}
    \setlength{\parskip}{6pt plus 2pt minus 1pt}}
}{% if KOMA class
  \KOMAoptions{parskip=half}}
\makeatother
\usepackage{xcolor}
\usepackage[margin=1in]{geometry}
\usepackage{color}
\usepackage{fancyvrb}
\newcommand{\VerbBar}{|}
\newcommand{\VERB}{\Verb[commandchars=\\\{\}]}
\DefineVerbatimEnvironment{Highlighting}{Verbatim}{commandchars=\\\{\}}
% Add ',fontsize=\small' for more characters per line
\usepackage{framed}
\definecolor{shadecolor}{RGB}{248,248,248}
\newenvironment{Shaded}{\begin{snugshade}}{\end{snugshade}}
\newcommand{\AlertTok}[1]{\textcolor[rgb]{0.94,0.16,0.16}{#1}}
\newcommand{\AnnotationTok}[1]{\textcolor[rgb]{0.56,0.35,0.01}{\textbf{\textit{#1}}}}
\newcommand{\AttributeTok}[1]{\textcolor[rgb]{0.13,0.29,0.53}{#1}}
\newcommand{\BaseNTok}[1]{\textcolor[rgb]{0.00,0.00,0.81}{#1}}
\newcommand{\BuiltInTok}[1]{#1}
\newcommand{\CharTok}[1]{\textcolor[rgb]{0.31,0.60,0.02}{#1}}
\newcommand{\CommentTok}[1]{\textcolor[rgb]{0.56,0.35,0.01}{\textit{#1}}}
\newcommand{\CommentVarTok}[1]{\textcolor[rgb]{0.56,0.35,0.01}{\textbf{\textit{#1}}}}
\newcommand{\ConstantTok}[1]{\textcolor[rgb]{0.56,0.35,0.01}{#1}}
\newcommand{\ControlFlowTok}[1]{\textcolor[rgb]{0.13,0.29,0.53}{\textbf{#1}}}
\newcommand{\DataTypeTok}[1]{\textcolor[rgb]{0.13,0.29,0.53}{#1}}
\newcommand{\DecValTok}[1]{\textcolor[rgb]{0.00,0.00,0.81}{#1}}
\newcommand{\DocumentationTok}[1]{\textcolor[rgb]{0.56,0.35,0.01}{\textbf{\textit{#1}}}}
\newcommand{\ErrorTok}[1]{\textcolor[rgb]{0.64,0.00,0.00}{\textbf{#1}}}
\newcommand{\ExtensionTok}[1]{#1}
\newcommand{\FloatTok}[1]{\textcolor[rgb]{0.00,0.00,0.81}{#1}}
\newcommand{\FunctionTok}[1]{\textcolor[rgb]{0.13,0.29,0.53}{\textbf{#1}}}
\newcommand{\ImportTok}[1]{#1}
\newcommand{\InformationTok}[1]{\textcolor[rgb]{0.56,0.35,0.01}{\textbf{\textit{#1}}}}
\newcommand{\KeywordTok}[1]{\textcolor[rgb]{0.13,0.29,0.53}{\textbf{#1}}}
\newcommand{\NormalTok}[1]{#1}
\newcommand{\OperatorTok}[1]{\textcolor[rgb]{0.81,0.36,0.00}{\textbf{#1}}}
\newcommand{\OtherTok}[1]{\textcolor[rgb]{0.56,0.35,0.01}{#1}}
\newcommand{\PreprocessorTok}[1]{\textcolor[rgb]{0.56,0.35,0.01}{\textit{#1}}}
\newcommand{\RegionMarkerTok}[1]{#1}
\newcommand{\SpecialCharTok}[1]{\textcolor[rgb]{0.81,0.36,0.00}{\textbf{#1}}}
\newcommand{\SpecialStringTok}[1]{\textcolor[rgb]{0.31,0.60,0.02}{#1}}
\newcommand{\StringTok}[1]{\textcolor[rgb]{0.31,0.60,0.02}{#1}}
\newcommand{\VariableTok}[1]{\textcolor[rgb]{0.00,0.00,0.00}{#1}}
\newcommand{\VerbatimStringTok}[1]{\textcolor[rgb]{0.31,0.60,0.02}{#1}}
\newcommand{\WarningTok}[1]{\textcolor[rgb]{0.56,0.35,0.01}{\textbf{\textit{#1}}}}
\usepackage{graphicx}
\makeatletter
\def\maxwidth{\ifdim\Gin@nat@width>\linewidth\linewidth\else\Gin@nat@width\fi}
\def\maxheight{\ifdim\Gin@nat@height>\textheight\textheight\else\Gin@nat@height\fi}
\makeatother
% Scale images if necessary, so that they will not overflow the page
% margins by default, and it is still possible to overwrite the defaults
% using explicit options in \includegraphics[width, height, ...]{}
\setkeys{Gin}{width=\maxwidth,height=\maxheight,keepaspectratio}
% Set default figure placement to htbp
\makeatletter
\def\fps@figure{htbp}
\makeatother
\setlength{\emergencystretch}{3em} % prevent overfull lines
\providecommand{\tightlist}{%
  \setlength{\itemsep}{0pt}\setlength{\parskip}{0pt}}
\setcounter{secnumdepth}{-\maxdimen} % remove section numbering
\ifLuaTeX
  \usepackage{selnolig}  % disable illegal ligatures
\fi
\IfFileExists{bookmark.sty}{\usepackage{bookmark}}{\usepackage{hyperref}}
\IfFileExists{xurl.sty}{\usepackage{xurl}}{} % add URL line breaks if available
\urlstyle{same}
\hypersetup{
  pdftitle={BeyondLinearityChapter},
  pdfauthor={Luca Marchiori},
  hidelinks,
  pdfcreator={LaTeX via pandoc}}

\title{BeyondLinearityChapter}
\author{Luca Marchiori}
\date{2024-05-11}

\begin{document}
\maketitle

\hypertarget{polynomial-regression}{%
\subsection{Polynomial Regression}\label{polynomial-regression}}

Polynomial regression extends the linear model by adding extra pre
dictors, obtained by raising each of the original predictors to a power.
This approach provides a simple way to provide a nonlinear fit to data.
For large enough degree d, a polynomial regression allows us to produce
an extremely non-linear curve. Generally speaking, it is unusual to use
d greater than 3 or 4 because for large values of d, the polynomial
curve can become overly flexible and can take on some very strange
shapes.

\[
Y = \beta_0 + \beta_1X + \beta_2X^2 + \beta_3X^3 + ... + \beta_dX^d + \epsilon
\] Notice that the coefficients can be easily estimated using least
squares linear regression because this is just a standard linear model
with predictors \(X, X^2, X^3, ..., X^d\).

\begin{Shaded}
\begin{Highlighting}[]
\CommentTok{\# Fit a linear model, using the lm() function, in order to predict wage using a fourth{-}degree polynomial in age}
\NormalTok{fit }\OtherTok{\textless{}{-}} \FunctionTok{lm}\NormalTok{(wage }\SpecialCharTok{\textasciitilde{}} \FunctionTok{poly}\NormalTok{(age , }\DecValTok{4}\NormalTok{), }\AttributeTok{data =}\NormalTok{ ds)}
\FunctionTok{summary}\NormalTok{(fit)}
\end{Highlighting}
\end{Shaded}

\begin{verbatim}
## 
## Call:
## lm(formula = wage ~ poly(age, 4), data = ds)
## 
## Residuals:
##     Min      1Q  Median      3Q     Max 
## -98.707 -24.626  -4.993  15.217 203.693 
## 
## Coefficients:
##                Estimate Std. Error t value Pr(>|t|)    
## (Intercept)    111.7036     0.7287 153.283  < 2e-16 ***
## poly(age, 4)1  447.0679    39.9148  11.201  < 2e-16 ***
## poly(age, 4)2 -478.3158    39.9148 -11.983  < 2e-16 ***
## poly(age, 4)3  125.5217    39.9148   3.145  0.00168 ** 
## poly(age, 4)4  -77.9112    39.9148  -1.952  0.05104 .  
## ---
## Signif. codes:  0 '***' 0.001 '**' 0.01 '*' 0.05 '.' 0.1 ' ' 1
## 
## Residual standard error: 39.91 on 2995 degrees of freedom
## Multiple R-squared:  0.08626,    Adjusted R-squared:  0.08504 
## F-statistic: 70.69 on 4 and 2995 DF,  p-value: < 2.2e-16
\end{verbatim}

\begin{Shaded}
\begin{Highlighting}[]
\CommentTok{\# Create a grid of values for age at which we want predictions}
\NormalTok{age.grid }\OtherTok{\textless{}{-}} \FunctionTok{seq}\NormalTok{(}\DecValTok{0}\NormalTok{, }\DecValTok{100}\NormalTok{)}

\CommentTok{\# Predict wage for all ages}
\NormalTok{preds }\OtherTok{\textless{}{-}} \FunctionTok{predict}\NormalTok{(fit , }\AttributeTok{newdata =} \FunctionTok{list}\NormalTok{(}\AttributeTok{age =}\NormalTok{ age.grid), }\AttributeTok{se =} \ConstantTok{TRUE}\NormalTok{)}

\CommentTok{\# Plot age and wage from the dataset}
\FunctionTok{plot}\NormalTok{(ds}\SpecialCharTok{$}\NormalTok{age , ds}\SpecialCharTok{$}\NormalTok{wage , }\AttributeTok{cex =}\NormalTok{ .}\DecValTok{5}\NormalTok{, }\AttributeTok{col =} \StringTok{"darkgrey"}\NormalTok{)}

\CommentTok{\# Predict wage for all ages}
\FunctionTok{lines}\NormalTok{(age.grid , preds}\SpecialCharTok{$}\NormalTok{fit , }\AttributeTok{lwd =} \DecValTok{2}\NormalTok{, }\AttributeTok{col =} \StringTok{"blue"}\NormalTok{)}
\end{Highlighting}
\end{Shaded}

\includegraphics{Beyond-Linearity---Chapter-7_files/figure-latex/unnamed-chunk-1-1.pdf}

\begin{Shaded}
\begin{Highlighting}[]
\CommentTok{\# Choose the polynomial degree using cross{-}validation}
\NormalTok{poly.mse}\OtherTok{=}\FunctionTok{c}\NormalTok{()}
\ControlFlowTok{for}\NormalTok{(degree }\ControlFlowTok{in} \DecValTok{1}\SpecialCharTok{:}\DecValTok{10}\NormalTok{)\{}
\NormalTok{  fit}\OtherTok{=}\FunctionTok{glm}\NormalTok{(wage}\SpecialCharTok{\textasciitilde{}}\FunctionTok{poly}\NormalTok{(age,degree,}\AttributeTok{raw=}\ConstantTok{TRUE}\NormalTok{),}\AttributeTok{data=}\NormalTok{Wage)}
\NormalTok{  cv }\OtherTok{=} \FunctionTok{cv.glm}\NormalTok{(fit,}\AttributeTok{data =}\NormalTok{ Wage,}\AttributeTok{K=}\DecValTok{10}\NormalTok{)}
\NormalTok{  MSE}\OtherTok{=}\NormalTok{cv}\SpecialCharTok{$}\NormalTok{delta[}\DecValTok{1}\NormalTok{]}
\NormalTok{  poly.mse}\OtherTok{=}\FunctionTok{c}\NormalTok{(poly.mse,MSE)}
\NormalTok{\}}

\NormalTok{degree.best }\OtherTok{=} \FunctionTok{which.min}\NormalTok{(poly.mse)}
\end{Highlighting}
\end{Shaded}

\hypertarget{step-functions}{%
\subsection{Step Functions}\label{step-functions}}

By using Step Functions, we break the range of X into bins, and fit a
different constant in each bin. Unless there are natural breakpoints in
the predictors, piece wise constant functions can miss the action. Step
function approaches are very popular in bio statistics and epidemiology,
among other disciplines. For example, 5-year age groups are often used
to define the bins.

Ww create cutpoints \(c_1, c_2, ..., c_K\) in the range of X, and then
construct \(K + 1\) new variables: \[
C_0(X) = I(X < c_1) \\
C_1(X) = I(c_1 \leq X < c_2) \\
C_2(X) = I(c_2 \leq X < c_3) \\
... \\
\] Note that \(I()\) is the indicator function that returns a 1 if the
condition is true, and 0 otherwise (these are sometimes called dummy
variables).

It is then possible to use least squares to fit a linear model using
\(C_0(X), C_1(X), ..., C_K(X)\) as predictors: \[
Y = \beta_0 + \beta_1C_1(x_i) + \beta_2C_2(x_i) + ... + \beta_KC_K(x_i) + \epsilon
\]

\hypertarget{basis-functions}{%
\subsection{Basis Functions}\label{basis-functions}}

Polynomial and piecewise-constant regression models are in fact special
cases of a basis function approach. \[
y_i = \beta_0 + \beta_1b_1(x_i) + \beta_2b_2(x_i) + ... + \beta_Kb_K(x_i) + \epsilon
\] where \(b_1(), b_2(), ..., b_K()\) are a set of functions typically
fixed and known, and are not estimated from the data.

\hypertarget{regression-splines}{%
\subsection{Regression Splines}\label{regression-splines}}

Piecewise polynomial regression involves fitting separate low-degree
polynomials over different regions of \(X\). The points where the
polynomial pieces meet are called knots.

For example, a piecewise cubic polynomial works by fitting a cubic
regression model of the form: \[
y_i = \beta_0 + \beta_1x_i + \beta_2x_i^2 + \beta_3x_i^3 + \epsilon
\] where \(\beta_0, \beta_1, \beta_2, \beta_3\) are different in
different parts of the range of \(X\). Using more knots leads to a more
flexible piecewise polynomial.

\[
y_i = \left\{\begin{matrix}
 & \\ 
 & 
\end{matrix}\right.
\]

To avoid having discontinuities in the first and second derivatives at
the knot locations, we can fit a piecewise polynomial under the
constraint that the fitted curve must be continuous at each knot. This
is known as a regression spline.

To add smoothness to the model at the knot locations, we can require
that not only the fitted curve is continuous at each knot, but also that
its first and second derivatives are continuous. This is known as a
smoothing spline. This regression splines can be somewhat complex since
we must fit a piecewise degree-d polynomial under the constraint that it
is continuous at all points, and that its first and second derivatives
are also continuous. To do this, we can use basis functions.

\[
y_i = \beta_0 + \beta_1b_1(x_i) + \beta_2b_2(x_i) + ... + \beta_Kb_K(x_i) + \epsilon
\] where \(b_1(), b_2(), ..., b_K()\) are basis functions and K are the
number of knots. The most direct way to represent a cubic spline is to
start off with a basis for a cubic polynomial (i.e.~\(x, x^2, x^3\)) and
then add one truncated power basis function per knot.

The regession spline is created by specifying a set of knots, producing
a set of basis functions and then using least squares to estimate the
spline coefficients.

Unfortunately, splines can have high variance at the outer range of the
predictors. A natural spline is a regression spline with additional
boundary constraints: the function is required to be linear at the
boundary (in the region where \(X\) is smaller than the smallest knot,
and in the region where \(X\) is larger than the largest knot). This
constraint makes the function more stable, and hence reduces the
variance, at the boundaries.

The regression spline is most flexible in regions that contain a lot of
knots, because in those regions the polynomial coefficients can change
rapidly. Hence, one option is to place more knots in places where we
feel the function might vary most rapidly, and to place fewer knots
where it seems more stable. While this option can work well, in practice
it is common to place knots in a uniform fashion. One way to do this is
to specify the desired degrees of freedom, and then have the software
automatically place the corresponding number of knots at uniform
quantiles of the data

A good way to choose the number of knots (degrees of freedom) is to use
cross-validation. We repeat cross validation for different numbers of
knots, and choose the number of knots that minimizes the RSS.

Regression splines often give superior results to polynomial regression.
This is because unlike polynomials, which must use a high degree to
produce flexible fits, splines introduce flexibility by increasing the
number of knots but keeping the degree fixed. Generally, this approach
produces more stable estimates. Splines also allow us to place more
knots, and hence flexibility, over regions where the function f seems to
be changing rapidly, and fewer knots where f appears more stable.

\hypertarget{smoothing-splines}{%
\subsection{Smoothing Splines}\label{smoothing-splines}}

\end{document}
